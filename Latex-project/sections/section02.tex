Chapter 2 is devoted to the theoretical part of this paper. The following assignments were the main interest for exploration in order to progress the current state of the work:
\begin{itemize}
    \item sound propagation concepts
    \item bubble backscattering, natural frequency, Thuraisingham and Anderson models
    \item multiple backscattering, other models implementation
    \item reconstruction of the bubble frequency response from the received signal
    \item beamforming
    \item cross-correlation, matched filtering
    \item sonar equation
    \item near and far field characteristics
\end{itemize}
\section{Sound propagation concepts}



\section{Bubble models}
% bubble backscattering, natural frequency, Thuraisingham and Anderson models
\subsection{Single bubble model}

There is a general formula of the gas bubble's rising velocity based on the bubble's volume, which allows us not to consider the bubble shape and its change during the rising process.


\begin{itemize}
    \item Anderson
    Anderson model provides a modal solution and allows to model the backscattering cross-section of a single bubble for $ka > 1$\cite{anderson_sound_2005}.

    \item Thuraisingham

    In this paper of Zhang et al. 2022\cite{zhang_efficient_2022} modal solution is used when ka > 1, and Thuraisingham solution is preferred for the ka<1, as the first doesn't consider the bubble damping effect.  Such approach can be used for our multiple compund bubble simulation.

    \item Church, Medwin,

    Andreeva is for fish.

\end{itemize}

\section{Multiple bubble model}

More bubbles are added

\subsection{Multiple scattering }
% Foldy paper equation explanation and concept of the multiple scattering
The concept lies in considering each bubble as a source when it scatters an incident wave. 
Therefore, each bubble will have an influence on another, and also receives a sum of all scatterrings from 
other bubbles. 

Bubbles have to be sufficiently far from each other in order not to be dependent on each others bubble frequency response.

\section{Reconstruction of the bubble frequency response}

It will allow demonstrating the ability to invert the model to find individual bubble contributions from compund analysis.
The following equation\ref{eqn:reconstruction_bubble} was the base of the calculations:

\begin{equation}\label{eqn:reconstruction_bubble}
    R = T \times H + N
\end{equation}
where R is a received signal, T is a transmitted signal by sonar, H is frequency response of the bubble, N is an added background noise.

\section{Beamforming}

% What is a beamforming? Why do we use it? How do we use it?
A beamforming is a way of processing the received or transmitted signal, so that we can direct the directed and amplified signal from teh projector with multiple receivers or transmitters, therefore improving the SNR and eliminating unwanted interfering signals. Also, it can be referred as a spatial filtering.

Withing the scope of the work we have used beamforming in the sonar environment. A conventional beamformer in the form of the delay-and-sum was used. 

\section{Sonar equation }
The Sonar equation is a fundamental tool in underwater acoustics used to predict the performance of sonar systems. 
The basic form of the Sonar equation is:
\begin{equation}
    SNR=SL-TL+TS-NL
\end{equation}

Where
SNR is the signal-to-noise ratio,
SL is the source level,
TL is the transmission loss,
TS is the target strength, and
NL is the noise level.

To implement the Sonar equation, one needs to calculate or estimate each of these parameters. 
Source level (SL) represents the acoustic power radiated by the sonar system, which can be determined based on the characteristics of the sonar transducer and the electrical input power. 
Transmission loss (TL) accounts for the attenuation of sound energy as it propagates through the water, considering factors such as spreading loss, absorption, and bottom and surface reflections. 
Target strength (TS) quantifies the amount of sound energy reflected or scattered by the target, which depends on its size, shape, and composition. 
Noise level (NL) encompasses all sources of ambient noise in the environment, including thermal noise, wind-generated noise, biological noise, and anthropogenic noise.

