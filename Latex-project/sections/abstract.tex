% \thispagestyle{plain}
% \begin{center}
%     \Large
%     \textbf{Acoustic Frequency Response Modelling of Multi-Bubble Compounds}
        
%     \vspace{0.4cm}
%     \large
%     % Thesis Subtitle
        
%     \vspace{0.4cm}
%     \textbf{Viktoriia Boichenko}
       
%     \vspace{0.9cm}
%     \textbf{Abstract}
% \end{center}


\Large
 \begin{center}
    Acoustic Frequency Response Modelling of Multi-Bubble Compounds

\hspace{10pt}

% Author names and affiliations
\large
Viktoriia Boichenko$^1$ \\

\hspace{10pt}

\small  
$^1$ vik.boichenko@gmail.com\\

\end{center}

\hspace{10pt}

\normalsize

This master's thesis focuses on developing a theoretical model for the acoustic frequency response of multi-bubble compounds, such as methane gas bubble seepage from natural reservoirs. The objective is to simulate and analyze the acoustic behavior of these compounds in water columns and to invert the models to identify individual bubble contributions. The research begins with a review of existing single- and multi-bubble acoustic response models, integrating these with effects like acoustic shadowing and bubble interactions. Initial models will be simple and progressively refined based on empirical data from previous studies and new measurements. Experiments will be conducted using a water tank to generate single and multi-bubble responses, with data used to validate and adjust the theoretical models. The final model will be incorporated into existing simulation software, such as KiRAT, to demonstrate its application as a digital twin for underwater column simulations and for generating SONAR data to train AI systems.