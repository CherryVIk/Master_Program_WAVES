Chapter 2 is devoted to the theoretical part of this paper. It presents main underwater acoustics concepts, basic definitions of bubble acoustics  and signal processing techniques. Also, it contains an introduction to bubble model and an explanation of the chosen approaches.

\section{Sound propagation concepts}
\begin{itemize}
    \item bubble backscattering, natural frequency, Thuraisingham and Anderson models
    \item multiple backscattering, other models implementation
    \item near and far field characteristics
\end{itemize}
The sound represents the transmission of the mechanical energy that  propagates through the medium and relies on its properties \cite[p.1]{leighton_acoustic_2012}.

Often used relation of the speed of the sound $c$ to angular frequency $\omega$ and a wavenumber $k$. 
 \[c = \omega / k\]

\subsubsection{Intensity} 
It is how much energy gets through the area perpendicular to the direction of the emitted wave \cite[p.18]{leighton_acoustic_2012}. The formula of acoustic intensity for a plane wave can be written in the following form:%a rate of energy in the wave crosses a unit area perpendicular to the direction of propagation
\[I = \frac{P_A^2}{2\rho c}\]

Intensity level is a measure over some area, which is a ratio of the sound intensity $I$ to some reference intensity $I_{ref}$ \cite[p.19]{leighton_acoustic_2012}:
\[IL = 10\log_{10}(\frac{I}{I_{ref}})\]
$I_{ref}=10^{-12} \text{W m}^{-2}$ in air.

As the pressure amplitude is proportional to the square root of intensity, we can write down sound pressure level in the form:
\[SPL = 20\log_{10}(\frac{P}{P_{ref}})\]

\subsubsection{Sound propagation loss} 

It is an amount of the transmitted sound signal received back by a receiver.

\subsubsection{Near- and far-field}
The phase difference that comes from the path difference is proportional to the ration of the path to the wavelength \cite[p.30]{leighton_acoustic_2012}.

\[D_\text{Near to far field} = L_s^2/\lambda\] 
where $L_s$ is the spacing between receivers
% find an image of hydrophones lobes with near and far field comparison


\textbf{Target strength} is how we measure the objects reflection of the signal. %my words , find justification


\section{Bubble models}
% bubble backscattering, natural frequency, Thuraisingham and Anderson models
\subsection{Single bubble model}

\begin{itemize}
    \item Anderson
    Anderson model provides a modal solution and allows to model the backscattering cross-section of a single bubble for $ka > 1$ \cite{anderson_sound_2005}.

    \item Thuraisingham

    In this paper of Zhang et al. 2022 \cite{zhang_efficient_2022} modal solution is used when ka > 1, and Thuraisingham solution is preferred for the ka<1, as the first doesn't consider the bubble damping effect.  Such approach can be used for our multiple compund bubble simulation.

    \item Church, Medwin,

    Andreeva is for fish.

\end{itemize}

% There is a general formula of the gas bubble's rising velocity based on the bubble's volume, which allows us not to consider the bubble shape and its change during the rising process????(whose citation)
\subsubsection{Anderson model}
The process of establishing the model for the acoustic frequency response of the bubble dates back to the paper of Anderson in 1950 \cite{anderson_sound_2005}. Initially, the theory of Rayleigh scattering has provided a big input into the scatterers which are comparable to a wavelength. While this work presented spheres with medium-like acoustic properties, and dimensions are as a couple  wavelengths. Using provided calculation we can obtain the pressure and the total energy in the scattered wave.

The following limitation of this model which should be considered is when ka approaches 0. That which means that the radius of sphere becomes less than a wavelength. For the cases when ka <1 or >1 there are no limitations.

This paper results can be used for comparison with other models available nowadays. The implementation in Matlab has allowed us to see how the peak at resonance frequency corresponds to the one of Thuraisingham model. However, at higher frequencies we can observes additional peaks, which indicate other modes of the fluid sphere, when ka is closer or over 1.
% We can express cross-section scattering using a relation of the reflectivity and a measure of the amount of power scatterer diverting from the original wave. That allows to obtain the total 

\subsubsection{Thuraisingham model}
The scattering cross-section $\sigma_s$ of a single bubble is a ratio of the energy loss averaged over time while being scattered from the bubble to the intensity of incident signal \cite[p.408]{thuraisingham_new_1997}. Thuraisingham formula for the scattering cross-section with the condition of a spherical pulsation of the bubble was presented in this form: 

\begin{equation}\label{eq:thuraisingham}
    \sigma_s=\frac{4\pi a^2}{\delta^2 + (\frac{\omega_r^2}{\omega^2}-1)^2}\frac{(\frac{\sin ka}{ka})^2}{1+(ka)^2
\end{equation}}




\subsubsection{Volume strength} 

\subsection{Multiple bubble model}


% More bubbles are added

\subsection{Multiple scattering }
\textcolor{red}{Foldy paper equation explanation and concept of the multiple scattering}

The main idea should be that you add up all the scattering from one scatterer, then from the other, till the end of all scatterers. When the signal is emitted, the scatterer removes from the wave a specific amount of flux which is equal to the its extintion cross section multiplied by flux per unit area. This amount is partially absorbed and scattered. 

But this image is applicable when amount of scatterers is sufficiently small, otherwise they start interacting with each other and no longer act as a scatter independently.

Hazard and Cassier paper provides a mathematical point of view at the multiple scattering with a justification of the Foldy-Lax model. This model help to get an approximation of scattered waves in a deterministic as well as in a random media. Also, local error estimates for circular objects in 2D problem were obtained in the scope of results of this paper.

- Foldy, Leslie L. ‘The Multiple Scattering of Waves. I. General Theory of Isotropic Scattering by Randomly Distributed Scatterers’. Physical Review 67, no. 3–4 (1 February 1945): 107–19. https://doi.org/10.1103/PhysRev.67.107.

- Multiple Scattering of Acoustic Waves by Small Sound-Soft Obstacles in Two Dimensions: Mathematical Justification of the Foldy–Lax Model, Hazard and Cassier


\section{Reconstruction of the bubble frequency response}

It will allow demonstrating the ability to invert the model to find individual bubble contributions from compund analysis.
The following equation\ref{eqn:reconstruction_bubble} was the base of the calculations:

\begin{equation}\label{eqn:reconstruction_bubble}
    R = T \times H + N
\end{equation}
where R is a received signal, T is a transmitted signal by sonar, H is frequency response of the bubble, N is an added background noise.

\section{Signal Processing}
\subsection{Beamforming}
\begin{itemize}
    \item reconstruction of the bubble frequency response from the received signal
    \item beamforming
    \item cross-correlation, matched filtering
    \item sonar equation
\end{itemize}

A beamforming is a way of processing the received or transmitted signal, so that we can direct the directed and amplified signal from teh projector with multiple receivers or transmitters, therefore improving the SNR and eliminating unwanted interfering signals. Also, it can be referred as a spatial filtering.

Withing the scope of the work we have used beamforming in the sonar environment. A conventional beamformer in the form of the delay-and-sum was used. 
\subsection{SONAR}
\subsection{Sonar equation }
The Sonar equation is a fundamental tool in underwater acoustics used to predict the performance of sonar systems. 
The basic form of the Sonar equation is:
\begin{equation}
    SNR=SL-TL+TS-NL
\end{equation}

Where
SNR is the signal-to-noise ratio,
SL is the source level,
TL is the transmission loss,
TS is the target strength, and
NL is the noise level.

To implement the Sonar equation, one needs to calculate or estimate each of these parameters. 
Source level (SL) represents the acoustic power radiated by the sonar system, which can be determined based on the characteristics of the sonar transducer and the electrical input power. 
Transmission loss (TL) accounts for the attenuation of sound energy as it propagates through the water, considering factors such as spreading loss, absorption, and bottom and surface reflections. 
Target strength (TS) quantifies the amount of sound energy reflected or scattered by the target, which depends on its size, shape, and composition. 
Noise level (NL) encompasses all sources of ambient noise in the environment, including thermal noise, wind-generated noise, biological noise, and anthropogenic noise.

