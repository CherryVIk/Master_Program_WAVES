%%% Introduction: presentation of the laboratory/structure, of your department, and the framework of the internship
The internship was performed as a part of the project within the scope of the hydroacoustic study field of the research instute GEOMAR of the Marine Geology, DeepSeaMonitoring department (DSM) with a collaboration of Technical faculty of the Christian-Albert University in Kiel. 

The framework within this internship was in a bubble acoustics, where a simulation work with an experimental measurements in the lab was done. 

Initially it started with an available literature review on the current topic. After that the simulation of the acoustic frequency response of the bubble with a SIMO-sonar system was performed using a Thuraisingham model and parameters described in Li et al paper \cite{li_broadband_2020}. Different levels of the complexity of the model were done progressively. Starting from the single static bubble modelling, folllowing by adding more bubbles, exploring them in the dynamic state and changing the distribution as well as the location of the bubbles. Simulations of the bubble flares were compared with the emperical data obtained after measurements.


% contextualize the study (project, industrial contract...)
% describe the working environment (company, lab,...) and position the study in this context
% allow you to understand the problematic treated, its constraints and the objectives sought,
% describe the work you have done and the methodology you have developed to answer it
% include a critical analysis of your results
% do not include a listing of programs; in case where you want to disseminate your code developed during the internship, make a link to a public repository

\subsection{Literature overview} 
%  of the single/multiple bubble modelling

In the paper of Foldy the mathematical background behind the scattering of the sphere has given a push to the further development of this property in bubbles.

The book of Ainslie has provided fundamentals for the sonar modelling which were essential for understanding while working with a sonar simulation. 




(Manasseh et al. 2004 \cite{manasseh_anisotropy_2004}) Bubbles produce an acoustic signal owing to compression of the gas in the bubble. The ‘spring’ of the compressible gas and the mass of liquid around the bubble create a natural oscillator, sending a pressure oscillation through the liquid and interacting with the neighbouring bubbles

(Zhang et al. 2022 \cite{zhang_efficient_2022} ) This paper provides a volume-scattering strength optimization model. This model allows to estimate the bubble size distribution. It provides a thorough explanation with the help of the case study experiment with a multibeam sonar to identify a bubble leakage in the sea.

It identifies three parameters: two in probability density function of gas leakage bubble sizes and the total number of bubbles inside the sample volume N0. Direct method was used for obtaining parameters.

(Li et al. 2020 \cite{li_broadband_2020}) Mentions theory regarding the bubble size distribution and provides the data for the backscattering cross-section of a single bubble. The model of the bubble plume’s acoustic backscattering consists of the model of the single bubble, distribution of size of the bubbles and computation of the volume scattering strength.

