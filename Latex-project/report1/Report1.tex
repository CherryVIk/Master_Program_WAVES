%Waves Program Master Degree Template
%Author: Viktoriia Boichenko
%vik.boichenko@gmail.com

% !TEX encoding = UTF-8 Unicode
% ! TEX program = lualatex

\documentclass[11pt]{article}

% Language setting
\usepackage[english]{babel}
\usepackage{fontspec}
% \setmainfont{Arial}
% Set page size and margins
\usepackage[letterpaper,top=2.54cm,bottom=2.54cm,left=3.27cm,right=3.27cm,marginparwidth=1.75cm]{geometry}
% Set spacing between lines and sections
\usepackage{titlesec}
\titlespacing*{\section}{0pt}{32pt}{20pt}
\titlespacing*{\subsection}{0pt}{16pt}{20pt}
\usepackage{setspace}
\onehalfspacing
%Adds space after paragraph
\usepackage{parskip}
\setlength{\parindent}{0pt}

% Useful packages
\usepackage{amsmath}
\usepackage{graphicx}
\usepackage[colorlinks=true, allcolors=blue]{hyperref}
\usepackage{units}
\usepackage{multirow}
\usepackage{caption}
\usepackage{csquotes}
\usepackage{float} %locate figures with [H]

%Import the bibliography file
\usepackage[backend=biber,style=ieee]{biblatex}
\begin{document}
% \addbibresource{../cites/bibliography.bib} 

%setting the cover page
\thispagestyle{empty}
% \begin{minipage}{0.96\linewidth}
% \includegraphics[width=1\linewidth]{figures/gandia-upv.png}
% \end{minipage}

\begin{center}
{\large\text{Erasmus Mundus WAVES}}   


\vspace*{4cm}
{\Large\textbf{Report \\ }}   


\vspace*{7cm}
% {\LARGE\text{Group N.ºN}}\\\vspace*{0.1cm}
 {\Large\text{Viktoriia Boichenko}} \\\vspace*{0.1cm}
 % {\Large\text{NAME SURNAME}}\\

\vspace*{2.3cm}
% {\large\text{Work developed in the scope of the course}\\\vspace*{0.1cm}
% \LARGE\textbf{COURSE NAME}}\\
\vspace*{0.2cm}
{\large\text{2023/2024}}   
\end{center}

\newpage

\tableofcontents

\section{Introduction}

The following assignments were the main interest for exploration in order to progress the current state of the work:

\begin{itemize}
    \item reconstruction of the bubble frequency response from the received signal
    \item simple version of the multiple scattering between bubbles
    \item research on the available mesurement procedures within the literature of the bubble experiments
    \item Wiener filter noise deconstruction
    \item sonar equation implementation
    % \item configuration of the latex environment in the vscode
\end{itemize}

\section{Theoretical part}

\subsection*{Reconstruction of the bubble frequency response from the received signal}

It will allow demonstrating the ability to invert the model to find individual bubble contributions from compund analysis.
The following equation \ref{eq:reconstruction_bubble} was the base of the calculations:
\begin{equation}\label{eq:reconstruction_bubble}
    R = T \times H + N
\end{equation}
where R is a received signal, T is a transmitted signal by sonar, H is frequency response of the bubble, N is an added background noise.


\subsection*{Simple version of the multiple scattering }

The concept lies in considering each bubble as a source when it scatters an incident wave. 
Therefore, each bubble will have an influence on another, and also receives a sum of all scatterrings from 
other bubbles. 

Bubbles have to be sufficiently far from each other in order not to be dependent on each others bubble frequency response.

\subsection*{Research on the available mesurement procedures within the literature of the bubble experiments}

\begin{itemize}
    \item  Sound signal: narrow band pulse, chirp, noise
    \item Sonar position: from top, vertical, horizontal
    \item Bubbles location: in the center of the experimental pool
   \item Bubble characteristics: emitting a single bubble of the specified radius; a row of bubbles; 
   creating a bubble flare;
\end{itemize}
Other things which are important for taking into account are:
\begin{itemize}
    \item Response of the transducer can influence the received signal
    \item Near field radiation implementation for measurements with a spherical radiation against the plane wave in a far field
\end{itemize}

Further set of things which are required to perform the experiment are the setup of the equipment required for performing our measurements. 
Essentially, it will include the sonar, a bubble generator, processing unit as a laptop/computer.

\subsection*{Wiener filter noise deconstruction}

It is one o the signal processing filters which allow us to denoise the signal. Main feature of the Wiener filter is 
that unlike simpler filters that may only suppress noise or attenuate certain frequency components, the Wiener filter operates on a statistical model of the signal and noise, effectively balancing the trade-off between signal fidelity and noise reduction.

subsection*{Sonar equation }
The Sonar equation is a fundamental tool in underwater acoustics used to predict the performance of sonar systems. 
The basic form of the Sonar equation is:
\begin{equation}
    SNR=SL-TL+TS-NL
\end{equation}

Where
SNR is the signal-to-noise ratio,
SL is the source level,
TL is the transmission loss,
TS is the target strength, and
NL is the noise level.

To implement the Sonar equation, one needs to calculate or estimate each of these parameters. 
Source level (SL) represents the acoustic power radiated by the sonar system, which can be determined based on the characteristics of the sonar transducer and the electrical input power. 
Transmission loss (TL) accounts for the attenuation of sound energy as it propagates through the water, considering factors such as spreading loss, absorption, and bottom and surface reflections. 
Target strength (TS) quantifies the amount of sound energy reflected or scattered by the target, which depends on its size, shape, and composition. 
Noise level (NL) encompasses all sources of ambient noise in the environment, including thermal noise, wind-generated noise, biological noise, and anthropogenic noise.


% The principle of the experiment to be conducted is … (a physical law applied to an analytical method) 
% the equation (mathematical form of the relationship between physical quantities), 
% the parameters (physical quantities kept constant during the experiment) 
% the variables of the experiment 
% the values of physical constants in their respective units of measurement or calculation diagram to aid in understanding

\section{Calculation}

\subsection*{Sonar equation implementation}
\begin{enumerate}
    \item Calculate the transmission loss from SONAR to 1m in front of target
    \item Calculate TL from 1m in front of target to target 
    \item Calculate TS from filtering with frequency response of target
    \item Add transmission loss from target to 1 m in front of target
    
    Combine these for total target strength
\end{enumerate}



% The following calculations were performed … (equation, substituting variables and consts, result in units)

% The mathematical development of calculations is presented below in symbolic form …

% Diagram of the experiment

% The following diagram is used to identify each element of the setup. It includes representations of measuring instruments, and the geometry of the setup. 

% List of the equipment used
% The used setup during the experiment was …

\section{Results}
% graphs, tables, or any other type of figures that address the  issue. 

% Figures should be accompanied by a title and an explanatory legend, similar to tables. 

% Don't forget to specify the nature and units. 
% Use a scale that maximizes the use of the surface area.


\section{Interpretation}

% The uncertainties associated with measurements were ..

% Influencing factors, relationships between variables on the results were …

% We have also performed the necessary calculations to validate our hypotheses which consisted of finding the values of …


% Following the lab instructions we have obtained …

\section{Conclusion}

% The hypotheses of … were validated. 

% The proposed solution to the posed problem was …

% The objective of the work .. was achieved …

% Comparing results from our simulation and calculations with the one obtained from literature are …

% The difference between them can be explained with … (due to a manipulation error, protocol design, or the experimental principle) 
\printbibliography
\end{document}
