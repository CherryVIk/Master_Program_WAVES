Chapter 2 is devoted to the theoretical part of this paper of undewater acoustics and signal processing essentials. It presents main concepts, basic definitions of bubble acoustics and signal processing techniques. An explanation of the chosen terms and approaches is provided below.

% \begin{itemize}
%     \item bubble backscattering, natural frequency, Thuraisingham and Anderson models
%     \item multiple backscattering, other models implementation
%     \item near and far field characteristics
% \end{itemize}
\section{Sound Propagation Concepts}

The sound represents the transmission of the mechanical energy that  propagates through the medium and relies on its properties \cite[p.1]{leighton_acoustic_2012}. The understanding presented below underwater acoustics basic terminology is essential for deeper comprehension of the processes that were involved in this research problem.
\subsubsection{Wavelength}
It defines the length of the wave over a unit period. It is an inverse of the wavenumber. The higher frequency, the shorter wavelength.
\begin{equation}
    \lambda = \frac{c}{f} \quad \frac{[\text{m/s}]}{[\text{1/s}]} = \text{m}
\end{equation}

\subsubsection{Speed of sound}
Generally, it is the speed of sound dependent on other parameter such as a temperature, a salinity and a depth, which make the speed of sound vary and refraction is appearing \cite[p. 28]{hodges_underwater_2011}. In order to simplify the simulation, the value of the speed of sound in a water was set at 1500 m/s. 
The equation \ref{eqn:c_water_relation} shows a relation of the speed of the sound $(c)$ to frequency $(\omega = 2\pi f)$ and a wavelength $(\lambda)$  
\begin{equation}
    c_{water} = f\lambda = \omega / k
    \label{eqn:c_water_relation}
\end{equation}
where $\omega$ is an angular frequency and $k=2\pi/\lambda$ is a wavenumber.
\subsection{Radiation of sound}

\subsubsection{Intensity} 
It is how much energy gets through the area perpendicular to the direction of the emitted wave \cite[p.18]{leighton_acoustic_2012}. The formula of acoustic intensity for a plane wave can be written in the following form:
\begin{equation}
    I = \frac{P_A^2}{2\rho c} = \frac{W}{4\pi s^2}
\end{equation}
Where $P_A$ is an acoustic pressure amplitude of the wave, $\rho$ is a density of the medium, $c$ is a speed of wave.

Where W is an energy, s is a distance from the source,
$4\pi s^2$ is a sphere surface.

\subsubsection{Intensity level}
Intensity level is a measure over some area, which is a ratio of the sound intensity $I$ to some reference intensity $I_{ref}$ \cite[p.19]{leighton_acoustic_2012}:
\begin{equation}
    IL = 10\log_{10}(\frac{I}{I_{ref}})    
\end{equation}
$I_{ref}=10^{-12}\text{W m}^{-2}$ in air.
As the pressure amplitude is proportional to the square root of intensity, we can write down sound pressure level in the form:
\begin{equation}
    SPL = 20\log_{10}(\frac{P}{P_{ref}})
\end{equation}

\subsubsection{Spherical spreading} 
The wave propagation with a $1/s^2$ dependance is defined as a spherical spreading.
% \tikz \draw (0,0) -- (2,1) -- (3,2);
% \begin{tikzpicture}
% \draw (0,0) circle (1);
% \draw (0,0) circle (1.5in);
% \end{tikzpicture}
\subsubsection{Attenuation  of sound}
The process of absorbtion and scattering of the sound can be defined as an attenuation. 

It is an amount of the transmitted sound signal received back by a receiver.


\section{Signal Processing Concepts}
% \begin{itemize}
    %     \item reconstruction of the bubble frequency response from the received signal
    %     \item beamforming
    %     \item cross-correlation, matched filtering
    %     \item sonar equation
    % \end{itemize}
    
\subsection{Beamforming}

A beamforming is a way of processing the received or transmitted signal, so that we can direct the directed and amplified signal from teh projector with multiple receivers or transmitters, therefore improving the SNR and eliminating unwanted interfering signals. Also, it can be referred as a spatial filtering.

Withing the scope of the work we have used beamforming in the sonar environment. A conventional beamformer in the form of the delay-and-sum was used. 

\subsection{SONAR}

\subsubsection{SIMO-sonar}

The general principle of the projector SIMO-sonar is in the following way:  the signal is emitted with a single transducer and receiving a signal back with several receivers, which in our case were hydrophones. Also, it can be considered with another interpretation. For example, we provide a single output, and obtain multiple input data.
    
The sonar type is the SIMO-sonar, as we needed to start with a simple model of the current research. The experiments with a MIMO-sonar can be implemented for further research therefore expanding the complexity and variability of usages of the developed model, as [this type provides an improved resolution and enhances the signal-to-noise ratio of the received signal] (http://jset.sasapublications.com/wp-content/uploads/2017/09/6702529.pdf).
    
	
\subsection{Sonar Equation }
The Sonar equation is a fundamental tool in underwater acoustics used to predict the performance of sonar systems. 
The basic form of the Sonar equation is:
\begin{equation}
    SNR=SL-TL+TS-NL
\end{equation}

Where
SNR is the signal-to-noise ratio,
SL is the source level,
TL is the transmission loss,
TS is the target strength, and
NL is the noise level.

To implement the Sonar equation, one needs to calculate or estimate each of these parameters. 
Source level (SL) represents the acoustic power radiated by the sonar system, which can be determined based on the characteristics of the sonar transducer and the electrical input power. 
Transmission loss (TL) accounts for the attenuation of sound energy as it propagates through the water, considering factors such as spreading loss, absorption, and bottom and surface reflections. 
Target strength (TS) quantifies the amount of sound energy reflected or scattered by the target, which depends on its size, shape, and composition. 
Noise level (NL) encompasses all sources of ambient noise in the environment, including thermal noise, wind-generated noise, biological noise, and anthropogenic noise.


\subsubsection{Near- and far-field}
The phase difference that comes from the path difference is proportional to the ration of the path to the wavelength \cite[p.30]{leighton_acoustic_2012}.

\[D_\text{Near to far field} = L_s^2/\lambda\] 
where $L_s$ is the spacing between receivers
In the near field the delay-and-sum equation for an array would look like:
\begin{equation}
    \label{eqn:far-field-DAS}
\end{equation}
% find an image of hydrophones lobes with near and far field comparison

\section{Reconstruction of the bubble frequency response}

It will allow demonstrating the ability to invert the model to find individual bubble contributions from compund analysis.
The following equation\ref{eqn:reconstruction_bubble} was the base of the calculations:

\begin{equation}\label{eqn:reconstruction_bubble}
    R = T \times H + N
\end{equation}
where R is a received signal, T is a transmitted signal by sonar, H is frequency response of the bubble, N is an added background noise.
